\documentclass[11pt]{article}
\renewcommand{\baselinestretch}{1.05}
\usepackage{amsmath,amsthm,verbatim,amssymb,amsfonts,amscd, graphicx}
\usepackage{graphics}
\topmargin0.0cm
\headheight0.0cm
\headsep0.0cm
\oddsidemargin0.0cm
\textheight23.0cm
\textwidth16.5cm
\footskip1.0cm
\theoremstyle{plain}
\newtheorem{theorem}{Theorem}
\newtheorem{corollary}{Corollary}
\newtheorem{lemma}{Lemma}
\newtheorem{proposition}{Proposition}
\newtheorem*{surfacecor}{Corollary 1}
\newtheorem{conjecture}{Conjecture} 
\newtheorem{question}{Question} 
\theoremstyle{definition}
\newtheorem{definition}{Definition}

\begin{document}
 
\title{Dynamical problems with variable mass}
\author{Kyle Herrity}
\maketitle

\begin{proposition}
 A spherical raindrop, starting from rest, falls under the influence of gravity. If it gathers in water vapor (assumed at rest) at a rate proportional to its surface, and if its initial radius is $0$, show that it falls with constant acceleration of $g/4$.
\end{proposition}
\begin{proof}
First some preliminaries. We can write the formula for the mass as a function of time of the raindrop.
\[ m = m(t) = \rho \frac{4}{3}\pi r(t)^3 \]
where $\rho$ is the density of the raindrop. Since we are told that the raindrop gathers water vapor at a rate proportional to the surface of the sphere, we have
\[ \frac{dm}{dt} = k 4\pi r(t)^2 \]
But
\begin{align} 
\frac{dm}{dt} &= \frac{dm}{dr}\frac{dr}{dt}\\
              &= \rho 4 \pi r^2 \frac{dr}{dt}\\
              &= k4\pi r^2
\end{align}
From which it follows that $\frac{dr}{dt} = \frac{k}{\rho}$. Therefore, $r(t) = \frac{k}{\rho}t$, since the raindrop starts with zero radius.
By Newton's 2nd law, we have
\[ (v + w)\frac{dm}{dt} + F = \frac{d}{dt}(mv) \]
which simplifies to 
\[  w\frac{dm}{dt} + F = m\frac{dv}{dt} \]
Here, $w$ is the velocity of the water vapor, relative to the raindrop, so $w = -v$. Also, the only external force acting on the raindrop is gravity, so that $F = -mg$. Thus, we have
\[  -v\frac{dm}{dt} - mg = m\frac{dv}{dt} \]
Rearranging, we get
\begin{align}
\frac{dv}{dt} &= -\frac{v}{m}\frac{dm}{dt} - g\\
              &= \frac{-v}{\rho(\frac{4}{3}\pi r^3)}k 4\pi r^2 - g\\
              &= \frac{-3vk}{\rho r}k - g\\
              &= \frac{-3v}{t} - g\\
\end{align}
So we have
\[ \frac{dv}{dt} + \frac{3v}{t} = -g \]
which is a standard first order linear differential equation, and we can use the integrating factor $\int \frac{3}{t} = 3\log(t)$.
Since $e^{3 log(t)} = t^3$,
\[ \frac{d}{dt}(t^3v) = -gt^3 \]
And after integrating both sides, we get
\[ t^3v = \int -gt^3 = -g\frac{t^4}{4} \]
So that
\[ v = \frac{-gt}{4} \]
So that the acceleration. $a$, is
\[ a = \frac{dv}{dt} = \frac{-g}{4} \]
\end{proof}
 
\end{document}
